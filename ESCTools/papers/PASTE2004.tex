% This used sigproc-sp.tex -FILE FOR V2.6SP OF ACM_PROC_ARTICLE-SP.CLS
% OCTOBER 2002
%
% ----------------------------------------------------------------------------------------------------------------
% This .tex file (and associated .cls V2.6SP) *DOES NOT* produce:
%       1) The Permission Statement
%       2) The Conference (location) Info information
%       3) The Copyright Line with ACM data
%       4) Page numbering
%
%  However, both the CopyrightYear (default to 2002) and the ACM Copyright Data
% (default to X-XXXXX-XX-X/XX/XX) can still be over-ridden by whatever the author
% inserts into the source .tex file.
% e.g.
% \CopyrightYear{2003} will cause 2003 to appear in the copyright line.
% \crdata{0-12345-67-8/90/12} will cause 0-12345-67-8/90/12 to appear in the copyright line.
%
% ---------------------------------------------------------------------------------------------------------------
% It is an example which *does* use the .bib file (from which the .bbl file
% is produced).
% REMEMBER HOWEVER: After having produced the .bbl file,
% and prior to final submission,
% you need to 'insert'  your .bbl file into your source .tex file so as to provide
% ONE 'self-contained' source file.
%
% Questions regarding SIGS should be sent to
% Adrienne Griscti ---> griscti@acm.org
%
% Questions/suggestions regarding the guidelines, .tex and .cls files, etc. to
% Gerald Murray ---> murray@acm.org
%
% For tracking purposes - this is V2.6SP - OCTOBER 2002

\documentclass{acm_proc_article-sp}
%\documentclass{sig-alternate}

\newif\ifpdf
\ifx\pdfoutput\undefined
   \pdffalse     % no PDFLaTeX
\else
   \pdfoutput=1  % PDFLaTeX
   \pdftrue
\fi

\ifpdf
\usepackage[pdftex,bookmarks=false,
            plainpages=false,naturalnames=true,
            colorlinks=true,pdfstartview=FitV,
            linkcolor=blue,citecolor=blue,urlcolor=blue]{hyperref}
\else
\usepackage[dvips]{hyperref}
\fi

\begin{document}
%
% --- Author Metadata here ---
\conferenceinfo{PASTE}{2004 Washington, D.C. USA}
%\setpagenumber{50}
\CopyrightYear{2004} % Allows default copyright year (2002) to be over-ridden - IF NEED BE.
%\crdata{}  % Allows default copyright data (X-XXXXX-XX-X/XX/XX) to be over-ridden.
% --- End of Author Metadata ---

\title{ESC/Java2: Uniting ESC/Java and JML}
\subtitle{Progress and issues in building and using ESC/Java2}
%
% You need the command \numberofauthors to handle the "boxing"
% and alignment of the authors under the title, and to add
% a section for authors number 4 through n.
%
% Up to the first three authors are aligned under the title;
% use the \alignauthor commands below to handle those names
% and affiliations. Add names, affiliations, addresses for
% additional authors as the argument to \additionalauthors;
% these will be set for you without further effort on your
% part as the last section in the body of your article BEFORE
% References or any Appendices.

\numberofauthors{2}
%
% You can go ahead and credit authors number 4+ here;
% their names will appear in a section called
% "Additional Authors" just before the Appendices
% (if there are any) or Bibliography (if there
% aren't)

% Put no more than the first THREE authors in the \author command
\author{
%
% The command \alignauthor (no curly braces needed) should
% precede each author name, affiliation/snail-mail address and
% e-mail address. Additionally, tag each line of
% affiliation/address with \affaddr, and tag the
%% e-mail address with \email.
\alignauthor David R. Cok\\
       \affaddr{457 Hillside Avenue}\\
       \affaddr{Rochester, NY}\\
       \affaddr{USA}\\
       \email{cok@frontiernet.net}
\alignauthor Joseph Kiniry\\
       \affaddr{Security of Systems Group}\\
       \affaddr{Computing Science Department}\\
       \affaddr{University of Nijmegen}\\
       \affaddr{Toernooiveld 1}\\
       \affaddr{6525 ED Nijmegen}\\
       \affaddr{The Netherlands}\\
       \email{kiniry@cs.kun.nl}
%\alignauthor Lars Th{\Large{\sf{\o}}}rv{$\ddot{\mbox{a}}$}ld\titlenote{This author is the
%one who did all the really hard work.}\\
%       \affaddr{The Th{\large{\sf{\o}}}rv{$\ddot{\mbox{a}}$}ld Group}\\
%       \affaddr{1 Th{\large{\sf{\o}}}rv{$\ddot{\mbox{a}}$}ld Circle}\\
%       \affaddr{Hekla, Iceland}\\
%      \email{larst@affiliation.org}
}
%\additionalauthors{}
\date{4 February 2004}

\newcommand{\myhref}[2]{\ifpdf\href{#1}{#2}\else\htmladdnormallinkfoot{#2}{#1}\fi}

\maketitle
\begin{abstract}
  The ESC/Java tool was a lauded advance in effective static checking
  of realistic Java programs, but has become out-of-date with respect
  to Java and the Java Modeling Language (JML).  The ESC/Java2
  project, whose progress is described in this paper, builds on the
  final release of ESC/Java from DEC/SRC in several ways.  It parses
  all of JML, thus can be used with the growing body of JML-annotated
  Java code; it has improved static checking capabilities, thus is
  becoming more complete and less unsound with each new release; and
  it has been designed, constructed, and documented in such a way as
  to improve the tool's availability to both users and researchers.
  It is intended that ESC/Java2 be used for further research in
  annotation and verification, for larger-scale case studies of
  annotation and verification, and for studies in programmer
  productivity that may result from integration with other tools that
  work with JML and Java.
\end{abstract}

\category{D.2.4}{Software Engineering}
                {Software/Program Verification}
                [Formal methods, Programming by Contract]
\category{F.3.1}{Logics and Meanings of Programs}
                {Specifying and Verifying and Reasoning about Programs}
                [assertions, invariants, logics of programs,
                pre- and post-conditions, specification techniques]

%% \terms{Delphi theory}

\keywords{JML, ESC/Java, ESC/Java2, static analysis, verification, annotation languages, program specification} 

%% I think that we currently have far too many bulleted lists.  They do 
%% not read well, nor are they space efficient.  I would hope we could
%% eliminate nearly all of them. -JRK

\section{Introduction}

The ESC/Java tool developed at DEC/SRC was a pioneering tool in the
application of static program analysis and verification technology to
annotated Java programs~\cite{ESCJava} [TODO-better ref].  The tool and its built-in
prover operated automatically with reasonable performance and needed
only program annotations against which to check a program's source
code.  The annotations needed were easily read, written and understood
by those familiar with Java and were mostly consistent with the syntax
and semantics of the separate Java Modeling Language (JML)
project~\cite{Leavens-etal00,jmlpapers}.  Consequently, the original
ESC/Java (hereafter called DEC/SRC ESC/Java) was a research success
and was also successfully used by other groups for a variety of case
studies (e.g.,~\cite{Hub03,HOP04}).

Its long-term utility, however, was lessened by a number of factors.
First, the match to JML was not complete and JML continued to evolve
as research on the needs of annotations for program checking advanced.
Second, some of the deficiencies of the annotation language used by
DEC/SRC ESC/Java, while appropriate in a research 
and demonstration mode, reduced the
overall usability of the tool.  For example, frame conditions were not
checked, but errors in frame conditions could cause the prover to
reach incorrect conclusions.  Also, the annotation language lacked the
ability to use methods in annotations, limiting the annotations to
statements only about low-level representations.  Finally, as
companies were bought and sold and research groups disbanded, there
was no continuing development of the original tool, making it
less useful as the Java language itself evolved and limiting the
amount of continuing support.

However, the initial positive experience prompted a vision for a
reasonably industrial-strength tool that would also be useful for
ongoing research in annotation and verification.  Thus, when the
source code for the DEC/SRC ESC/Java tool was made available for
further research and experimentation, the authors of this paper began
the ESC/Java2 project to evolve the DEC/SRC ESC/Java tool.  That
effort has the following goals:
(1) to make the source consistent with the current version of Java;
(2) to fully parse the current version of JML;
(3) to check as much of the JML annotation language as feasible,
consistent with the original engineering goals of DEC/SRC ESC/Java
(usability at the expense of full completeness and soundness); 
(4) to package the tool in a way that enables easy application in a
variety of environments, consistent with the licensing provisions of
the source code release;
and
(5) as a long-term goal and if appropriate, to update the related
tools that use the same code base (Calvin, Houdini, RCC) and to
integrate with other JML-based tools.  Success in the above will
enable testing the utility of the tool to improve programmer
productivity on a broader range of significant bodies of Java source,
and to use the tool as a basis for research in unexplored aspects of
annotation and static program analysis.
  
We currently have an alpha version of ESC/Java2
[[TODO - the web reference doesn't get printed]]
\myhref{http://www.niii.kun.nl/ita/sos/projects/escframe.html}{available
  on the web} and encourage experimentation and feedback.  The source
code itself is available (and additional contributors are welcome)
subject to fairly open licensing provisions.  The discussion below of various
features of JML and ESC/Java2 is necessarily brief; more detail is
available in the implementation notes that are part of an ESC/Java2 release.

The subsequent sections will discuss the principal changes made to DEC/SRC ESC/Java in creating ESC/Java2 including the extensions to static checking, the backwards incompatibilities introduced, the outstanding features of JML yet to be implemented, unresolved semantic issues in JML, the direction of the ongoing work in this project, and the implications of this 
work in the area of programming productivity.  We fully acknowledge
that the on-going work described here builds on the very substantial
and innovative work to produce ESC/Java and the Simplify prover in the
first place and on the effort to define the Java Modeling Language.


\section{Changes to ESC/Java}

The work on ESC/Java2 required embellishments of the original DEC/SRC
ESC/Java in a number of areas.  Here we present the most significant
of these.
\subsection{Java 1.4}
The original work was performed in 1998 and 2000, and Java has evolved
since then.\footnote{In fact, Java 1.5 went beta recently.  No work has begun on
  parsing or statically checking Java 1.5 code.  Interested parties
  are welcome to contact the authors with regard to this work.}  The
principal addition required by Java 1.4 is support for the Java {\tt
  assert} statement.  

JML itself contains a similar assert statement.  Hence, the user may make a choice between
two behaviors.  A Java assert statement may be interpreted simply as another language
feature whose behavior is to be modeled.  The corresponding behavior is to raise an
 \texttt{AssertionError} exception under appropriate circumstances.  Alternatively, 
a Java assert statement may be interpreted as a JML assert statement.  In this case, the
static checker will report a warning if the assertion predicate cannot be established to be true.

\subsection{Current JML}
The Java Modeling Language is a research project in itself; hence the
JML syntax and semantics are evolving and are somewhat of a moving
target (and there is as yet no complete reference manual).  However,
the core language is reasonably stable.  The following are key
additions that have been implemented:
\setlength{\partopsep}{0in}\setlength{\parskip}{0in}\setlength{\itemsep}{0in}\setlength{\topsep}{0in}
\begin{itemize}
\setlength{\partopsep}{0in}\setlength{\parskip}{0in}\setlength{\itemsep}{0in}\setlength{\topsep}{0in}
\item parsing of all of current JML, even if the constructs are
  ignored with respect to typechecking or verification;
\item support for refinement files, which allow specifications to be supplied in files separate from the source code or in the absence of source code;
\item heavyweight annotations, which allow some degree of modularity and nesting;
\item inheritance of annotations and of \texttt{non\_null}
  modifiers that is consistent with the behavioral inheritance of JML;
\item support for datagroups and \texttt{in} and \texttt{maps} clauses, which provides a sound framework for reasoning about the combination of frame conditions and subtyping;
\item model import statements and model fields, routines, and types, which allow abstraction 
and modularity in writing specifications;
\item auto model import of the \texttt{org.jmlspecs.lang} package, similar to Java's auto import of \texttt{java.lang};
\item enlarging the use and correcting the handling of scope of ghost fields, so that the syntactic behavior 
of annotation fields matches that of Java and other JML tools;
\item generalizing the use of \texttt{\char'134 old}, \texttt{set} statements and local ghost variables, to provide more flexibility in writing specifications;
\item introduction of the \texttt{constraint}, \texttt{represents}, \texttt{field}, \texttt{method},
\texttt{constructor}, 
\texttt{\char'134 not\_modified}, \texttt{forall}, \texttt{old}, \texttt{instance}, \texttt{pure} keywords as defined in JML;
\item and, consistency in the format of annotations, in order to match the language handled by other JML tools.
\end{itemize}
In addition, the differences between DEC/SRC ESC/Java and JML noted in the JML
Reference Manual have been resolved.
% I don't think that there was ever a paper published summarizing 
% these differences. -JRK

\subsection{New verification checks}
Though all of JML is parsed, not all of it is currently checked.
Static checking of the following features has been added to that performed by
DEC/SRC ESC/Java.
\setlength{\partopsep}{0in}\setlength{\parskip}{0in}\setlength{\itemsep}{0in}\setlength{\topsep}{0in}
\begin{itemize}
\setlength{\partopsep}{0in}\setlength{\parskip}{0in}\setlength{\itemsep}{0in}\setlength{\topsep}{0in}
\item \texttt{constraint} and \texttt{initially} clauses and \texttt{\char'134 not\_modified} expressions: These JML features allow more flexibility and conciseness in writing annotations.
\item frame conditions, including the \texttt{in} and \texttt{maps} clauses:  This addition corrects
most of an unsoundess present in DEC/SRC ESC/Java and also adds additional checking to detect
incorrect frame conditions.
\item model fields, \texttt{represents} clauses: These allow modularity and abstraction in
writing specifications, particularly for interfaces and abstract classes \cite{Cheon-etal03}.
\item annotations that include (pure) routine calls and dynamic allocations: These also contribute
to abstraction and allow much more natural and readable specifications.
\end{itemize}


\subsection{Backwards incompatibilities}
The DEC/SRC ESC/Java specification language and JML arose separately.  There
was some initial but incomplete work to unify the two.  The ESC/Java2
project intends to have the tool reflect JML as precisely as possible.
In some cases, discussion about differences resulted in changes to JML.  In
a few cases, some backwards incompatibilities in DEC/SRC ESC/Java were
introduced.  The principal such incompatibilities are these:
\setlength{\partopsep}{0in}\setlength{\parskip}{0in}\setlength{\itemsep}{0in}\setlength{\topsep}{0in}
\begin{itemize}
\setlength{\partopsep}{0in}\setlength{\parskip}{0in}\setlength{\itemsep}{0in}\setlength{\topsep}{0in}
\item The semantics of inheritance of specification clauses and of
  \texttt{non\_null} modifiers was modified to match that defined by
  JML, since the work on JML resulted in an interpretation consistent
  with behavioral subtyping.  This also changed the usage and
  semantics of \texttt{also}.
\item The specification \texttt{modifies \char'134 everything;} is now the default
  modifies clause.
\item The syntax and semantics of \texttt{initially}, \texttt{readable\_if} 
 and \texttt{monitored\_by} have changed.
\item ESC/Java2 forbids bodies of (non-model) routines to be present
  in non-Java specification files.
\end{itemize}

\section{Unimplemented Aspects of JML}

Though the core is well-supported, there are several features of JML
which are parsed and ignored, some of them experimental or not yet
endowed with a clear semantics, and some in the process of being implemented.
  The features that are currently ignored are the following:
\setlength{\partopsep}{0in}\setlength{\parskip}{0in}\setlength{\itemsep}{0in}\setlength{\topsep}{0in}
\begin{itemize}
\setlength{\partopsep}{0in}\setlength{\parskip}{0in}\setlength{\itemsep}{0in}\setlength{\topsep}{0in}
\item checking of access modifiers on annotations and of the
 \texttt{strictfp}, \texttt{volatile},
  \texttt{transient} and \texttt{weakly} modifiers;
\item the \texttt{diverges}, \texttt{hence\_by},
  \texttt{code\_contract}, \texttt{when}, and \texttt{measured\_by}
  clauses;
\item the annotations within \texttt{implies\_that} and {\tt
    for\_example} sections;
\item some of the semantics associated with initialization (prior to
  construction);
\item multi-threading support beyond that already provided in DEC/SRC ESC/\-Java;
\item serialization;
\item annotations regarding space and time consumption;
\item full support of recursive \texttt{maps} declarations;
\item model programs;
\item equivalence of \texttt{\char'134 TYPE} and \texttt{java.lang.Class};
\item some aspects of store-ref expressions;
\item verification of anonymous and block-level classes;
\item verification of set comprehension and some forms of quantified
  expressions;
\item implementation of \texttt{modifies \char'134 everything;} within
  the body of routines.
\end{itemize}

\section{Unresolved semantic issues}

The work on ESC/Java2 has been useful in exposing and resolving
semantic issues in JML.  Since ESC/Java2 is built on a different
source code base than other JML tools, differences of interpretation
in both syntax and semantics arise on occasion.  These are generally
resolved and documented via mailing list discussions\footnote{See \texttt{jmlspecs-interest@lists.sourceforge.net}, 
\texttt{jmlspecs-developers@lists.sourceforge.net}, and
\texttt{jmlspecs-escjava@lists.sourceforge.net}}
by interested
parties.  There are, however, still unresolved issues, most of which
are the subject of ongoing research.
\setlength{\partopsep}{0in}\setlength{\parskip}{0in}\setlength{\itemsep}{0in}\setlength{\topsep}{0in}
\begin{itemize}
\setlength{\partopsep}{0in}\setlength{\parskip}{0in}\setlength{\itemsep}{0in}\setlength{\topsep}{0in}
\item \textit{pure routines}:  It is convenient and modular to use model and Java methods within
annotations.  The semantics of such use is clearer and simpler if such routines are {\em pure}, that is, they do not have side-effects.  This is important when evaluating annotations during
execution since the checking of specifications should not affect the running of the program being
checked.  Side-effects would also complicate static reasoning, though in that case one could 
in theory define a set of specifications for a routine that is somewhat different than its actual effect, but is
appropriate to the verification task.   However, some side-effects are
always present, such as changes to the stack or heap or external effects such as the passage of time.  Some are often overlooked but can be consequential, such as locking a monitor. 
 Others the programmer may
see as innocuous, benevolent side-effects, such as maintaining a private cached value or logging debugging information in an output file.
An interpretation of the combination of purity and benevolent (or ignorable)
  side-effects that is suitable for both static and run-time checking and is convenient and intuitive
for users is not yet available.
\item \textit{exceptions in pure expressions}:  The expressions used in annotations must not have
side-effects, but they may still throw exceptions.  In that case the result is ill-defined.  A
semantics that is suitable for both run-time checking and static verification needs to be 
established.
\item \textit{initialization}:  The authors are not aware of any detailed work on specifying the initialization
of classes and objects in the context of JML.  This includes providing syntax and semantics for the
Java initialization blocks, the \texttt{initializer} and \texttt{static\_initializer} keywords in
JML, and formalizing the rules about order of initialization of classes and object fields in Java.
\item \textit{helper annotations}:  Both JML and ESC/Java define a \texttt{helper} keyword that enables 
routines to be written that need not obey class invariants.  However the permitted uses of this
keyword vary.
\item \textit{\texttt{measured\_by} clause}:  The \texttt{measured\_by} clause was defined in JML to
provide a clause to assist in proving termination.  It was removed when its semantics
proved troublesome.  However, the problem of establishing appropriate annotations that
can assist a static checker in ensuring termination remains.
\item \textit{\texttt{code\_contract} clauses}:  The \texttt{code\_contract} section of a specification and its
\texttt{accessible} and \texttt{callable} clauses are intended to assist in creating subclasses 
without reference to source code \cite{Ruby-Leavens00}.  However, the tools to automatically
generate these clauses and studies of their practicality in soundly verifying correct behavior have
not been completed.
\item \textit{model programs}:  Model programs have been proposed as an alternate means to write
specifications in JML, however no tool support exists as yet.
\item \textit{datagroups}:  The \texttt{in} and \texttt{maps} clauses and the datagroup syntax are designed
to allow the specification of frame conditions in a sound way that is extensible by derived classes. 
We do not yet have experience with the interaction among datagroups, the syntax for designating store locations, and either reasoning about recursive data structures or checking them at 
run-time.
\item \textit{\texttt{hence\_by}, \texttt{weakly}}:  These annotations have had some theoretical discussion but
there is as yet little tool support or practical experience with their use.
\item \textit{unbounded
  arithmetic}:  Chalin~\cite{Chalin03} has proposed syntax and semantics to enable specifiers to utilize unbounded arithmetic in a safe way within annotations.  Tool support and experience with
these concepts is in progress.  Axioms and proof procedures will be needed to support this work
in static checkers.
\end{itemize}

\section{Usage experience to date}
The SoS group at the University of Nijmegen, along with other members
of the \myhref{http://www.verificard.org/}{European VerifiCard
  Project}, has used DEC/SRC ESC/Java for several projects.  For
example, Hubbers, Oostdijk, and Poll have performed verifications of
Smart Card applets using several tools, including DEC/SRC
ESC/Java~\cite{HOP04}.  Hubbers has also taken initial steps
integrating several JML-based tools~\cite{Hub03}.

These and other VerifiCard projects relied upon the specifications of
the Java Card 2.1 API written and verified by Erik Poll and Hans
Meijer~\cite{MeijerPoll01}.  This specification originally came in two
forms: one ``complete'' specification that used JML models,
heavyweight specifications, and refinements that was used with the
core JML tools, and another ``lightweight'' specification that was
meant to be used with DEC/SRC ESC/Java and other verification tools
like Jack, Krakatoa, and the Loop tool~\cite{a,b,c}.  [[TODO -fix these]]

Writing, verifying, and maintaining these two specifications was
sometimes a troublesome experience.  Because of limitations of various
tools which depended upon the specifications, several alternate forms
of specifications were sometimes required.  Additionally, it was
sometimes the case that the alternate forms were not equivalent nor
had obvious logical relationships between them.

This experience was one of the motivators for the SoS group's support
of this work on ESC/Java2.  Now that multiple tools are available that
fully cover the JML language, specification reuse is rising and
negative maintenance issues are decreasing.  

Early evidence for the success this transition is beginning to appear.

First, the specifications of a small case
study~\cite{BreunesseJacobsBerg02} were updated and re-verified by one
of the authors using ESC/Java2.  The original work depended upon
lightweight JML specifications of core Java Card classes and the
verification was performed with DEC/SRC ESC/Java and the Loop tool.
On the other hand, the re-verification effort used the full
``heavyweight'' Java and Java Card specifications and was accomplished
in a single afternoon.

Second, several members of the SoS group are contributing to updating
the ``heavyweight'' JML specifications of the Java Card API.  As a
part of this work, the Gemplus Electronic Purse case study, which has
been verified partially with DEC/SRC ESC/Java~\cite{CatanoHuisman02}
and completely with the Loop tool~\cite{JoachimsThesis?}, is being
re-verified. [[ TODO - with Esc/Java2??]]

TODO - citation above

--general opinion of work--
--transition of pov--

[I will add a one paragraph discussion here of our experiences in
the SoS group with using ESC/Java for verification of core JDK classes
and Smart Card applets. -JRK]

\section{Ongoing work}
The work on ESC/Java2 to date has resulted in a number of software
alpha releases and the work is continuing on a number of fronts.

\paragraph*{Language Issues.} Two obvious and related ongoing tasks are
the completion of additional features of JML, accompanied in some
cases by additional research to clarify the semantics and usability of
outstanding features of JML.  Usage of JML is now broad enough that
accompanying formal reference documentation would be valuable.  As
tools such as ESC/Java2 become more widely used, users will also
appreciate attention to performance, to the clarity of errors and
warnings, and to the overall user experience such tools provide.

\paragraph*{Case Studies.} The current implementation supports the
static checking of a stable core of JML.  With this initial
implementation of frame condition checking and of model fields,
represents clauses and use of routine calls in annotations, ESC/Java2
can now be used on complex and abstract specifications of larger
bodies of software.  Consequently, there is a considerable need for
good experimental usage studies that confirm that this core of JML is
useful in annotations and that the operation of ESC/Java2 (and
Simplify) on that core is correct and valuable.

\paragraph*{Verification Logic.} The logic into which Java and JML are
embedded in both ESC/Java and ESC/Java2 is, by design and admission of the original
authors, neither complete or fully sound.  This was the result of an
engineering judgement in favor of performance and usability.  Research
that studies expanded and larger use cases may show whether this
design decision is generally useful in practical static checking or
whether a fuller and more complicated state-based logic is required
for useful results to be obtained. 

A related issue is the balance between automated and manual proof construction.
Use of verification logics will likely be limited to narrow specialties as long as proof
construction is a major component of the overall programming task.  Thus automation
is essential, though it is expected that full automation is infeasible.  The degree of
automation achievable will continue to be a research question. 
However, we believe that broad adoption of automated tools for program checking 
will require that users only need interact at high levels of proof construction.

\paragraph*{User Feedback.} The purpose of using theorem provers for
static analysis, runtime assertion checking, or model checking is to
find errors and thereby improve the correctness of the resulting
software.  Thus the orientation of a tool must be towards effectively
finding {\em and interpreting} examples of incorrect behavior.  A
complaint (e.g. \cite{GroceVisser03}) in using such tools is that it is difficult
to determine a root cause from the counterexample information provided
by the tool, whether it is a failed proof or an invalid test or
execution history.  The ESC/Java project implemented some work towards
appropriately pruning and interpreting counterexample and trace information, but
there remains room for improvement.

\paragraph*{Tool Integration.} Finally, though not part of this specific
project, there is another set of beneficial integration work to be
done.

There are now a number of tools, generated by different research
groups, that interact with JML~\cite{Burdy-etal03}.  Among them, the
JML compiler jmlc~\cite{Cheon-Leavens02b} dynamically checks
specifications given test data or in actual operation; 
Bogor \cite{Bogor03} does so using model checking;
the ESC/Java2 tool statically checks specifications;
Daikon [[TODO - ref]] generates specifications from execution histories (and demonstrated an
integrated working style
with DEC/SRC ESC/Java~\cite{NimmerErnst01});
the JML Unit Test tool jmlunit~\cite{Cheon-Leavens02} generates test
cases from specifications.
 
A productive programmer's working environment for a large-scale
project that uses these tools would need the tools to be integrated in
a way that their outputs and inputs seamlessly inter-communicate.  A
programmer using the tools would naturally move among the various
tasks of designing, writing,
testing, annotating, verifying and debugging, all the while reading, writing
and checking specifications.  Design, specicfications and code might all
be built up incrementally.  Thus the tools would
need to be integrated in a way that allows efficient and iterative
behavior.

For example, once specifications generated by a Daikon-like tool are
generalized and incorporated into a method's annotations, repeated
running of Daikon will only produce results already implied by the
existing specifications.  That may be cause to ignore those subsumed
results until changes elsewhere in the program or new test data show
that the annotations are incorrect.  In that case, the integrated tool,
when rerunning the specification-generation step, would ignore any
generated specifications that are implied by those already present.
On the other hand, if the
specifications generated by analyzing the execution history are
implied by but narrower than the written specifications, one might
conclude that either the test data is too limited or the written
specifications (at least in this usage) are overly broad.  In that mode,
the user would want a clear diagnosis of the circumstances which
are implied by the specifications but not exercised during execution. 

%% I am personally not a fan of conclusions that do not say anything 
%% additional (i.e., real implications of the paper).  Also, for a 
%% paper this short, a conclusion might just be wasted space.  If we 
%% can find something to say here that is, in fact, implicatory, they 
%% I'd be fore it. -JRK

\section{Conclusion}

This work on implementing ESC/Java2 and evaluating its use is in progress, so a
set of data-driven conclusions is premature.  Instead, let us posit a hypothesis in the form of a vision.

One can observe this work in tool creation and evaluation from a number 
of perspectives.  Certainly such work creates working prototypes that test in practice
theories of programming and specification language semantics.
It also exercises and validates ideas in automated logical reasoning.
We prefer to use the viewpoint of programming productivity,
particularly given the industrial working environment of the first author.  In that
context we observe 
the existence and general use across multiple research groups of the combination of various tools that support
using JML with Java programs; this suggests to us
 that the syntax and semantics of the core of JML are 
sufficiently useful and natural to provide a basis for future wider use.
With respect to logical reasoning, a useful degree of automation is achievable in 
at least some aspects of
static checking tasks; removing the details of proof construction from a programmer's
tasks is essential to larger scale acceptance of such tools.

However, the surrounding issues are as relevant to programmer acceptance and
productivity.  Tools must have intuitive and unsurprising behavior. They must be efficient
in elapsed run-time, but also in the time needed to interpret and act upon the results.
They must integrate well with other tools of the same family and with the commonly used
programmer's working environments. 

There is progress on enough of the above vectors that one might well be optimistic about the
eventual success of the enterprise as a whole.  After all, the goal need not be fully
automated verification of an
arbitrary computer program.  Reflect that a computer-produced proof
of a mathematical conjecture that cannot be understood at least in its broad outlines by
mathematicians leaves those mathematicians unsatisfied and unsettled with respect to the
result.  Similarly, we expect that ``verifications" of programs whose overall
design is incomprehensible
to readers of the program (not to mention its author) will not engender much confidence
in the result.  If programming is writing for others and we expect that the authors could 
explain their programs to their colleagues, we may well have a chance at being able to explain those programs to a computer.


%\large\textbf{TODO:}\normalsize Perhaps discuss the fact that a
%verifying compiler might be closer than some notable computer
%scientists think~\cite{Hoare03}?  Discuss how far we think that we can
%take this application of basic logics, first-order provers, and
%related technologies.  Also reinforce the changing balance between
%automated and interactive proving, discussed in the usage experience
%section above. -JRK

%ACKNOWLEDGMENTS are optional
\section{Acknowledgments}
The authors would like to acknowledge both the work of the team that
generated ESC/Java as well as Gary Leavens and collaborators at Iowa
State University who generated JML.  In addition, Leavens and students
engaged in and helped resolve syntactic and semantic issues in JML
raised by the work on ESC/Java2.  These teams provided the twin
foundations on which the current work is built.  In addition, the many
other research groups that use and critique both JML and ESC/Java2
have provided a research environment in which the work described here
is useful.  Joseph Kiniry is supported by the NWO Pionier research
project on Program Security and Correctness and the VerifiCard
research project.

%
% The following two commands are all you need in the
% initial runs of your .tex file to
% produce the bibliography for the citations in your paper.
\bibliographystyle{abbrv}
\bibliography{PASTE2004}  
% To create the self-contained file - comment out the line above and include the
% contents of the .bbl file here

% You must have a proper ".bib" file
%  and remember to run:
% latex bibtex latex latex
% to resolve all references
%
% ACM needs 'a single self-contained file'!
%
%APPENDICES are optional
\balancecolumns

% That's all folks!
\end{document}
