\documentclass{sig-alternate}
\usepackage{url}
\usepackage{verbatim}
\usepackage{graphicx}
%% boxed-figure-new.tex
%
% formatting boxed displays

%% some parameters
\newlength{\BFIGURESKIP}
\setlength{\BFIGURESKIP}{3mm}
\newlength{\BFIGURELINEHEIGHT}
\setlength{\BFIGURELINEHEIGHT}{0.3pt}

%% Use the following for a line at the top of a figure,
%% it is raised by \BFIGURESKIP
%% The \par at the end seems to be needed for postscript figures that follow.
\newcommand{\BTOPFIGURELINE}{%
{\rule[\BFIGURESKIP]{\columnwidth}{\BFIGURELINEHEIGHT}}\par%
}

%% A line, with not raised
\newcommand{\BFIGURELINE}{
{\rule{\columnwidth}{\BFIGURELINEHEIGHT}}
% \begin{flushleft}\rule{\columnwidth}{\BFIGURELINEHEIGHT}\end{flushleft}%
}

%% regular figures, the width of the current column of text
\newenvironment{BFIGURE}{
\begin{figure}
\BFIGURELINE
}{%
\par\vskip-\parskip
\BFIGURELINE%
\end{figure}
}

%% regular figures, the width of two column of text, at the top of the page
\newenvironment{BFIGUREDB}{
\begin{figure*}[t!]
}{%
\par\vskip-\parskip
\BFULLFIGURELINE%
\end{figure*}
}

%% figures on a page by themselves
\newenvironment{BFIGUREP}{
\begin{figure}[p!]
\BFIGURELINE
}{%
\par\vskip-\parskip
\BFIGURELINE%
\end{figure}
}

%% figures at the top of the page
\newenvironment{BFIGURET}{
\begin{figure}[t!]
\BFIGURELINE
}{%
\par\vskip-\parskip
\BFIGURELINE%
\end{figure}
}

%% figures at the bottom of the page
\newenvironment{BFIGUREB}{
\begin{figure}[b!]
\BTOPFIGURELINE
}{%
\par\vskip-\parskip
\BFIGURELINE%
\end{figure}
}

%% A line, with not raised
\newcommand{\BFULLFIGURELINE}{
{\rule{\textwidth}{\BFIGURELINEHEIGHT}}
% \begin{flushleft}\rule{\textwidth}{\BFIGURELINEHEIGHT}\end{flushleft}%
}

%% full page figures, the width of the whole page
\newenvironment{BFIGURE*}{
\begin{figure*}
\BFULLFIGURELINE
}{%
\par\vskip-\parskip
\BFULLFIGURELINE%
\end{figure*}
}

%% full page figures, on a page by themselves
\newenvironment{BFIGUREP*}{
\begin{figure*}[p!]
\BFULLFIGURELINE
}{%
\par\vskip-\parskip
\BFULLFIGURELINE%
\end{figure*}
}

%% full page figures, at the top of a page
\newenvironment{BFIGURET*}{
\begin{figure*}[t!]
\BFULLFIGURELINE
}{%
\par\vskip-\parskip
\BFULLFIGURELINE%
\end{figure*}
}

%% another way to do horizontal lines
\newcommand{\UNSPACEFORBOX}{\vspace{-\BFIGURESKIP}}
\newcommand{\HLINE}{\UNSPACEFORBOX\BFIGURELINE\UNSPACEFORBOX}

%% end of boxed-figure-new.tex

%{obey}

% \myobeycr is same as \obeycr, but doesn't do a \@gobblecr.
{\catcode`\^^M=13 \gdef\myobeycr{\catcode`\^^M=13 \def^^M{\\}}%
\gdef\restorecr{\catcode`\^^M=5 }}

% \obeytabs
{\catcode`\^^I=13 \gdef\obeytabs{\catcode`\^^I=13 \def^^I{\hbox{\hskip 4em}}}}

% \obeyspaces
{\obeyspaces\gdef {\hbox{\hskip0.5em}}}

% environment for displayed text, indented, obeys cr, tab, spaces
\newenvironment{obeyDisplay}{%
\samepage%
\begin{list}{}{}\item\obeyspaces\obeytabs\obeycr}{%
\end{list}}


\newif\ifpdf
\ifx\pdfoutput\undefined
   \pdffalse     % no PDFLaTeX
\else
  \pdfoutput=1  % PDFLaTeX
   \pdftrue
\fi

\ifpdf
\usepackage[pdftex,bookmarks=false,
            plainpages=false,naturalnames=true,
            colorlinks=true,pdfstartview=FitV,
            linkcolor=blue,citecolor=blue,urlcolor=blue]{hyperref}
\else
\usepackage[dvips]{hyperref}
\fi

% macros for program code
%%\input{program-code}

\newcommand{\INCLUDECOMMENT}[1]{\textrm{$\langle$ {#1} $\rangle$}}

\begin{document}

\renewcommand{\floatpagefraction}{0.9} 
\renewcommand{\textfraction}{0.1}

%%% The following for acm style
\conferenceinfo{FTFJP}{Oslo, Norway}
\CopyrightYear{2004} 

\title{Reasoning with specifications containing method calls in JML and first-order provers}
%\subtitle{}

\numberofauthors{2}
\author{
\alignauthor David R. Cok\\
       \affaddr{457 Hillside Avenue}\\
       \affaddr{Rochester, NY   14610}\\
       \affaddr{USA}\\
       \email{cok@frontiernet.net}
\alignauthor Joseph Kiniry\\
       \affaddr{Security of Systems Group}\\
       \affaddr{Computing Science Department}\\
       \affaddr{University of Nijmegen}\\
       \affaddr{Toernooiveld 1}\\
       \affaddr{6525 ED Nijmegen}\\
       \affaddr{The Netherlands}\\
       \email{kiniry@cs.kun.nl}
}

\date{30 March 2004}

\toappear{Submitted to the Formal Techniques for Java-like Programs 
workshop at ECOOP 2004}
%%% The following for llncs style
%\pagestyle{headings}  % switches on printing of running heads
%\addtocmark{XXXXXXXXX} % additional mark in the TOC
%

%\mainmatter
%\title{Reasoning with specifications containing method calls in JML}
%\titlerunning{YYYYYYY}


%\author{David R. Cok\inst{1} \and Joseph Kiniry\inst{2}}

%\institute{457 Hillside Avenue\\
%       			Rochester, NY   14610\\
%      			USA \\
%       		{\tt cok@frontiernet.net} \\
%	\and
%		Security of Systems Group\\
 %     		Computing Science Department\\
 %      		University of Nijmegen\\
%       		Toernooiveld 1\\
%       		6525 ED Nijmegen\\
%       		The Netherlands\\
%       		{\tt kiniry@cs.kun.nl}
%}

%\authorrunning{Cok and Kiniry}

%%%%%%%%%%%%%%%%%%%%%%%%%%%%%%%%%%%%%%%%%

\date{27 March 2004}

\newcommand{\myhref}[2]{\ifpdf\href{#1}{#2}\else\htmladdnormallinkfoot{#2}{#1}\fi}

\maketitle
\begin{abstract}
Program verification requires translating the constructs of a specification 
into a suitable logic for an underlying theorem prover.   Providing modularity
and comprehensibility by allowing 
annotations containing method calls and other abstractions is as important in 
specifications as it is in the program itself.  However, method invocations do not
map neatly into the first-order logics that are often used for assuring the correctness
of specifications.  The ESC/Java2 tool
has been able to achieve a practical solution to this difficulty within the design constraints
of its parent tool, ESC/Java.  Furthermore, the techniques used are applicable to other
specification constructs such as quantifiers and model variables.

\end{abstract}

%%\category{D.2.4}{Software Engineering}
%%                {Software/Program Verification}
%%                [Formal methods, Programming by Contract]
%%\category{F.3.1}{Logics and Meanings of Programs}
%%                {Specifying and Verifying and Reasoning about Programs}
 %%               [assertions, invariants, logics of programs,
%%                pre- and post-conditions, specification techniques]


%%\keywords{JML, ESC/Java, ESC/Java2, static analysis, verification, annotation languages, program specification} 

\section{Introduction}

Research in program verification is proceeding along many fronts: 
increasing sophistication of annotation languages, increasing
understanding of the nuances of programming language semantics as they
relate to verification challenges, increasing sophistication of 
provers, and increasing experience in using implemented systems on
industrial-scale problems.  Early treatments ignored complicated
aspects such as object references and method calls, which are
essential in application to existing software.  Allowing
the use of methods in specifications provides a level of abstraction 
and conciseness that
promotes reading, writing and understanding specifications and will 
likely assist in their automated verification as well.  

One such system,
ESC/Java \cite{ESCJava} was successful at checking annotated programs, but its
absence of method calls in specifications limited those annotations to
low-level representations.  That system translated Java programs into
an untyped first-order language.  In this paper we discuss the implementation
of an extension to ESC/Java that allows the use of routine calls in
specifications.  With that accomplished, several other programming
language constructs can also be handled by the underlying prover.
The approach described here is applicable to any source code translator
that must interface with a prover that operates in a generic first-order 
logic (as opposed to a logic specifically designed to handle programming language
constructs).

Sections2, 3, and 4 describe the Java Modeling
Language, the ESC/Java2 project, and the Simplify prover; these systems
are the context in which the solution described here was implemented 
and tested.  Section 5 describes our solution for translating method calls.
Applications to other specification language features are described in section 6.
Section 7 describes some related work,
and we end with some outstanding issues in section 8.


\section{The Java Modeling Language}

The Java Modeling Language (JML) has by now been described in several publications
\cite{jmlpapers,Leavens-Baker-Ruby99b,Leavens-Baker-Ruby02} and we will not repeat that full description here.  This language is suitable for
describing the issues at hand and is used in the implementation of the solution.
 The advances described in this paper
can be illustrated using simple invariants and lightweight preconditions,
frame conditions and postconditions.  
A class with examples of these annotations is shown in Fig.~\ref{fig:SS}.

\begin{BFIGURE}
\verbatiminput{SortedSequence.java}
\caption{The specification and code for the class \texttt{SortedSequence}.  This class maintains a Bag of integer values, always returning the largest value.  For demonstration purposes, 
the sorting of values is performed only on demand.}
\label{fig:SS}
\end{BFIGURE}

Specifications are included in the text of a Java program by placing them in 
specially formatted comments; these comments are either single-line comments
beginning with  \texttt{//@} or are multi-line comments beginning with  \texttt{/*@}.
Some specifications, such as the \texttt{invariant} clause
in Fig.~\ref{fig:SS},  are properties of the class.
It contains a predicate that is expected to be satisfied in 
each 'visible' state, that is, in the program state just after execution of constructors and
just before and after execution of methods of the object.  In addition each routine may
have a set of specifications: a \texttt{requires} clause 
stipulates a precondition that must be satisfied in the program state just prior to a call
of the method with which it is associated; an \texttt{ensures} clause stipulates the 
postcondition that must be satisfied in the program state just after method execution;
the \texttt{assignable} clause lists the only memory locations that may be assigned to (and hence 
potentially modified) in the course of execution of the method.  The \texttt{pure} modifier on the \texttt{sort} method states that that method does not assign to any non-local memory locations
and does not have any side-effects (that is, it does not change any observable aspect of 
the program state).

The syntax of the specification predicates follows Java closely.  It excludes any operations 
that have side effects, such as the increment operator
 \texttt{++}.  Other operations, such as arithmetic and 
comparison operators, have the same syntax and semantics as in Java.  In particular, 
predicates may include method calls, if those methods are designated \texttt{pure},
indicating that the method has no side-effects.  Purity is
required because annotations must be statements
about the annotated program and may not modify its state.  This is particularly the case for
tools which execute and check annotations at run-time.

\section{ESC/Java2}

The ESC/Java tool~\cite{Flanagan-etal02} developed at DEC/SRC 
(and subsequently at Compaq) was a pioneering tool in the
application of static program analysis and verification technology to
annotated Java programs.  The tool and its built-in
prover operated automatically with reasonable performance and needed
only program annotations against which to check a program's source
code.  The annotations needed were easily read, written and understood
by those familiar with Java and were partially consistent with the syntax
and semantics of the separate Java Modeling Language (JML)
project~\cite{jmlpapers,Leavens-etal00}.  Consequently, the original
ESC/Java was a research success
and was also successfully used by other groups for a variety of case
studies (e.g.,~\cite{Hub03,HOP04}).

The ESC/Java2 project \cite{CokKiniry04} was begun to extend ESC/Java
and its long-term utility by addressing a number of issues.
\begin{itemize}
\setlength{\itemsep}{0in}
\setlength{\parsep}{0in}
\item As companies were bought and sold and research groups
disbanded, there was no continuing development or support of 
ESC/Java and the tool was untouched for over two years and its
source code was not available.  The ESC/Java2 project has established
an open-source repository for the tool, brought the code up to the current
definition of Java, and provided a degree of ongoing development and
maintenance.
\item  ESC/Java only partially supported JML and JML continued to evolve as
research on the needs of annotations for program checking advanced.
This unavoidable divergence of specification languages made writing,
verifying, and maintaining specifications of non-trivial APIs
troublesome.
Additionally, JML has grown significantly in popularity.  A number of
tools have been developed that use JML, representing the work of several
groups~\cite{jmlpapers,Burdy-etal03,Leavens-etal00,NimmerErnst01,Bogor03,BergJ01}.
Thus, many new research tools worked well with ``modern'' JML, but
ESC/Java did not.  ESC/Java2 has evolved from ESC/Java to be
consistent with the current definition of JML.
\item Finally, and most relevant to this paper,
not all of the specification constructs of JML were checked by ESC/Java.  For
example, frame conditions were not checked, but errors in frame
conditions could cause the prover to reach incorrect conclusions.
Also, the annotation language lacked the ability to use methods in
annotations, limiting the annotations to statements only about
low-level representations.
\end{itemize}

The engineering goals of ESC/Java were to be automatic and useful in finding bugs
and violations of program specifications.  It was not designed to be complete nor
entirely sound.  ESC/Java2 has continued that spirit, though some unsound aspects have
been corrected.

\section{SIMPLIFY and the target verification condition language}

As is customary in this field, ESC/Java2 (following the parent tool ESC/Java) translates program source code into guarded commands, then into a single-assignment representation, and 
finally
into verification conditions (VCs).  These conditions are passed to an accompanying prover which
judges them to be valid or invalid and may produce counterexamples to demonstrate
invalidity.  With an appropriately general definition of the syntax and semantics of the verification
condition language, various provers could be used and compared.  ESC/Java2 uses the
Simplify \cite{nelson80,simplify} prover, accepting verification conditions expressed in a
first-order logic (including universal and existential quantification) with equality and untyped total
functions, extended with a simple theory of arithmetic.

Briefly, the input language for VCs consists of the following:
\begin{itemize}
\item a VC is a logical formula, constructed using the conventional constants and connectors
\texttt{TRUE}, \texttt{FALSE}, \texttt{AND}, \texttt{OR}, \texttt{NOT}, \texttt{IMPLIES},
\texttt{IFF}, \texttt{FORALL}, \texttt{EXISTS}, \texttt{EQ}, and \texttt{NEQ}.  Here \texttt{EQ} and
\texttt{NEQ} are equality and the negation of equality among pairs of terms.
\item Terms are either constants, variables (as used in quantification), or function terms.
\item Function terms consist of function symbols and zero or more arguments, which are also
terms.  Function symbols are typically uninterpreted, but may have a meaning supplied by
one of the supporting theories.  For example, numeric literals will have an interpreted meaning
to Simplify's theory of linear inequalities over rational variables.
\item Constants are uninterpreted symbols or ground constants of one of the theories.
\end{itemize}

Different provers will implement different sets of decision procedures on this input language.
Simplify includes four interacting subtheories that cooperate to assess the satisfiability of a 
set of input formulas:
\begin{itemize}
\item an e-graph sub-theory, which reasons about the equality (as in \texttt{EQ}) of sets of
(uninterpreted) terms;
\item a simplex sub-theory, which reasons about arithmetic equalities and inequalities (including less-than and greater-than) over linear combinations
of terms;
\item an ordering sub-theory, which reasons about partial orders among terms;
\item a matching sub-theory, which reasons about quantified formulas.
\end{itemize}

In the context of a programming language, the prover has built-in knowledge of numerical and
boolean values and operators on those values.  A base set of axioms describes the behavior 
of types and the subtype relationship and of arrays.  Object identity corresponds to a simple
equality relationship among untyped, uninterpreted constants.  Object fields are modeled as
arrays: a field named \texttt{f} is considered an array, and a field reference \texttt{o.f} is
translated as the array reference \texttt{f[o]}.

\section{Translating method calls}

\subsection{Issues with method calls within annotations}

The problem at hand is to translate expressions containing method calls into the target logic described above.  An assertion in the target logic concerning some equivalent of a method call is of no value in reasoning, however, unless there are other logical assertions that provide some 
information to the prover about that translation of the method call.  That information resides in
the specifications of the method being called.  Thus, we need to translate the specifications
of the called method
in a manner similar to that used to translate the calling expression.

In some cases there exists an expression whose value is the result of the method call.  For
instance, if a method's specifications have a postcondition of the form
\begin{center}\texttt{ensures \char'134 result == ... ; }\end{center}
then one could extract such an expression, at least under the preconditions for which the
postcondition holds.  (The JML symbol \texttt{\char'134 result} represents the result returned by a method.)  
That expression could then be substituted for the method call itself,
after appropriate substitution of actual for formal arguments.  This procedure does not work
in general however.  There may not be such an expression available.  There may be more than
one such
expression available, requiring a prescient choice of the best one to substitute for the method call.
Alternatively, the expression being substituted may contain other method calls that
will themselves require substitution; the substitution procedure may not terminate if there is
any recursive use of method calls in the annotations.  Even without recursion, the depth of
rewriting can create very large verification conditions (easily consuming 256M on the ordinary
but realistic sets of specifications contained in the JML library, in our experiments).

However, since the logic does allow terms containing uninterpreted functions, defining a new
uninterpreted function for each method is not a difficulty.  A naive translation of methods to
function terms would translate a method call of \texttt{sort()} into a function term with no
arguments, namely, \texttt{(sort)}.  This procedure encounters the following complications.
\begin{itemize}
\item
As will be obvious to any Java programmer, the argument list of the method does not 
actually contain all of the relevant arguments to the method.  If the method is not static, the
argument list must include the receiver object (\texttt{this}) as well.  Not quite as obviously,
functions representing constructors of inner classes must also include a reference to the
enclosing class as an argument.  Thus \texttt{sort()} must be translated as \texttt{(sort this)}.
This allows a natural distinction between method calls \texttt{sort()}, \texttt{a.sort()}, and
\texttt{b.sort()}.  These are translated into the terms \texttt{(sort this)}, \texttt{(sort a')}, and
\texttt{(sort b')} (here \texttt{a'} and \texttt{b'} are the translations into terms of
the programming language expressions \texttt{a} and \texttt{b}, respectively).  
Aliasing is naturally handled as well, since if it is established, for example, that
\texttt{(EQ this a')}, then \texttt{(EQ (sort this) (sort a'))} will immediately follow.

\item
Secondly, the method implementation may depend on other static or instance fields
of the receiving object that are not listed in the argument list.  The values of instance fields
may be considered to be implicitly included via the \texttt{this} argument, but their values then
depend on the current program state.  In the context of Fig.~\ref{fig:SS}, 
the value of the \texttt{sort} term
will depend on the values of \texttt{this.n} and the values of the elements of \texttt{this.seq}.

\item
Most importantly, the semantics of equality among function terms is not appropriate to the
reference semantics of an object-oriented language such as Java.  Two function terms in the
logic are equal if they have the same function symbol, the same number of arguments, and the
arguments are pairwise equal.  This definition of equality is fine for the immutable
values of Java's primitive types, but not for reference values.  Reference values to the
same object in different program
states will test equal even though their internal states may be different.  For example,
\texttt{this.sort()} will have a different value in the pre-state and the post-state of a call of the 
\texttt{push} method.  However, in the logic \texttt{(sort this)} will be the same term, and therefore
equal, in those two states, since there is no knowledge of a program state in the underlying
logic or embedded in the reference value of \texttt{this} in the logic.
\end{itemize}

\subsection{The translation procedure}

Two options were explored to overcome the difficulty presented in the previous section.

First, one could expand the argument list of a method to include directly all the fields of all
of the objects (including \texttt{this}) in the argument list.  This can create very long
argument lists.  In general, it is not even possible to do this syntactically since the reference
arguments may refer to recursive data structures.

The more useful alternative is to include as an argument of the method a value indicating the
program state in which the method is being evaluated.  A different state constant is defined
for the pre-state and the post-state if the method is not \texttt{pure}; a common state
constant can be used if the method is \texttt{pure}.  This state constant is uninterpreted.
That is we do not use it in any context other than to distinguish different program states 
in method calls.
Thus we keep the relative simplicity of the single-assignment translation into a first-order
logic without embedding a full state-based representation of the programming language into the 
logic.
This may limit the conclusions that may be drawn about relationships among 
program states, but has
been simple to apply and thus far suitably effective in practice.

The translation, then, consists of the following steps,
\begin{itemize}
\item Select a unique identifier for each method declaration in the program (e.g. by using its
textual location in a specific file).
\item Define a unique state constant (distinct from all other constants) for each unique program
state within the method.  If the method is \texttt{pure}, the pre-state constant and the post-state
constant may be identical.  A new state is created after every operation with a side-effect.  In
practice state constants are only needed for those program points where an annotation with
a method call (or other construct that is translated with a state argument) occurs.
\item Where a method call is used in an annotation expression, translate that method call
into the Simplify input language as a function call.  Use the unique identifier for the static
type of the method as the function name.  Include as arguments the translations of (a) the
state constant for this pre- or post-state, (b) the receiver object (if the method is not static),
and (c) each of the usual arguments of the method.
\item The specifications of the called method must be turned into assumptions.  The 
specifications must first be desugared by combining preconditions and postconditions
into stand-alone implications of the form (\cite{Raghavan-Leavens03} describes the details): \\
{\centering \texttt{ensures } {\em precondition}\texttt{ ==> }{\em postcondition}; }.\\
Any instance of \texttt{\char'134 result} is replaced by an instance of the method call, with
formal names for its arguments.  The expression is enclosed in a universal quantification over
its formal parameters.  Thus (in a class named \texttt{Z})\\
\texttt{requires i != 0;}\\
\texttt{ensures \char'134 result == i+1; }\\
\texttt{public int next(int i);}\\
in a state with state constant \texttt{stateX} creates the axiom \\
\texttt{ (\char'134 forall Z object; (\char'134 forall int i;  i != 0 ==> \\
             next(stateX,object,i) == i + 1; ))} \\
Since we do not extract all values (e.g. of fields) out of a program state, we do not quantify over
the state constant.  Instead we repeat the assumption above in each context where the method
is called and translate any free variables in the context of that call.
\item JML allows annotations to appear in the body of a method as well, using \texttt{assert}
statements for example.  These are translated in the same way as postconditions; they simply
use a different state constant.
\end{itemize}

[[[ Experience; example of what is not possible to conclude ]]]

\section{Application to other \\ annotation constructs}

With translation of method calls to an underlying first-order logic enabled in ESC/Java2, 
several other specification constructs can be readily translated and used in static checking as 
well.  These are described briefly in this section.
 
\subsection{Constructor calls}

Constructor calls can be treated as calls of static methods.  That is, they do not depend on an
implict \texttt{this} argument, though, if they are constructors of a java inner class, they will depend 
on an implicit argument representing an instance of the enclosing class.  Since some of the 
arguments may be reference values, the translated function must also have a state constant
as an argument.  The axioms about the constructed value are formed from the postconditions
of the constructor declaration.

Constructors are different than methods in that they dynamically allocate new objects on the 
heap.  Thus the result is a reference value not equal to any previous reference value.  ESC/Java2
(following ESC/Java) provides axioms concerning allocation that ensure this behavior, but 
those are beyond the scope of this paper.

\subsection{Array allocation}

Translating expressions such as \texttt{new int[9]} that allocate new arrays is quite straightforward.
These expressions do not depend on the current state nor on any implicit receiver argument.
Consequently a function whose arguments are simply the type and number of the array 
elements is all that is needed.
Just as for constructors, axioms regarding allocation are required, 
so that the value produced by a new array
expression is known to be different than any previously produced reference value.

A new array expression has no explicit pre-conditions or post-conditions, but there are some implicit axioms:
the result of the new array allocation is not null, is freshly allocated,
 and has a specific dynamic type, 
a specific length, and each
element is equal to a zero-equivalent value (e.g. integer 0, boolean \texttt{false}, or
reference \texttt{null}).

\subsection{Quantified expressions}

JML includes in its expression syntax various generalized quantified expressions.  Besides
the usual universal (\texttt{forall}) and existential (\texttt{exists}) quantified expressions, JML
also defines the quantifiers \texttt{\char'134 min}, \texttt{\char'134 max}, \texttt{\char'134 numof},
\texttt{\char'134 sum}, and \texttt{\char'134 prod}.  For example, the value of the expression
\begin{center} \texttt{(\char'134 min int i; i <= 0 \&\& i < 10; p(i))} \end{center}
is the smallest value of \texttt{p(i)} for \texttt{i} in the given range.  Similarly, \texttt{\char'134 max}
produces the largest of a given set of \texttt{int} values;  \texttt{\char'134 numof} yields the number
of true values in a set of boolean values; the value of the
 \texttt{\char'134 sum} and \texttt{\char'134 prod}
expressions is the sum and product, respectively, of the set of given values.

The translation of each of these consists of syntactically replacing the expression with a
function call (unique to that syntactic location).  Implicit receiver and state arguments are needed
as described previously.  If the quantifier is within the scope of another quantified
expression, there will also need to be function arguments for any bound variable used
in the replaced expression.

One must also introduce axioms concerning the value of this introduced function, 
corresponding to the value of the original quantified expression.  For example, the
axioms associated with
\begin{center} ( \texttt{\char'134 min} {\em decl}; {\em range-expr}; {\em expr}) \end{center}
are 
\begin{center} ( \texttt{\char'134 exists} {\em decl}; {\em range-expr}) \texttt{==>} \\
( \texttt{\char'134 exists} {\em decl}; {\em range-expr}; \texttt{MIN() ==} {\em expr}) \end{center}
and
\begin{center} ( \texttt{\char'134 exists} {\em decl}; {\em range-expr}) \texttt{==>} \\
( \texttt{\char'134 forall} {\em decl}; {\em range-expr}; \texttt{MIN() <=} {\em expr}), \end{center}
where \texttt{MIN()} is replaced by the actual skolem function call expression.


\subsection{Model variables}

Model variables are declarations of fields within annotation comments that are not treated
as fields of the object.  Rather a model variable is given a representation, typically in 
terms of the internal state of the object.  The model variable may be used in annotations
as an abstract representation, or {\em model}, of some quantity related to the object at hand.
For example, [[[ ... provide example ]]]

Model variables in JML may have a functional representation, a predicate representation, or 
no representation at all.  Work on translation of model variables in JML for the LOOP tool 
occurred concurrently with the work in this paper and is discussed in 
\cite{BreunessePoll03}.  We came to
a similar solution and offer some additional observations here.

\subsubsection {Functional representations}

A functional representation is denoted in JML by the syntax
\begin{center} //@ \texttt{represents x <- }{\em ... expression ... } ; \end{center}
There is a specific value, provided by the expression, for the model variable (in a given program 
state).   This expression could be simply substituted for occurences of the model variable, as
stated by \cite{BreunessePoll03}.  However, this is successful only in simple cases.  If there
is heavy use of model variables, the nested substitutions can be quite deep.  Furthermore,
JML allows multiple redundant representations, requiring a choice of which to use.  Finally,
direct or mutual recursion would prohibit simple substitution.  In our implementation in
ESC/Java2, we found that representing functional model variables by method
calls was the better
solution.

\subsubsection{Predicate representations}

JML also allows the values of model variables to be specified with a predicate representation:
\begin{center} \texttt{represents x \char'134 such\_that }{\em ... predicate ... } ; \end{center}
In this case the predicate does not necessarily give an executable expression for the
model variable and may not even constrain the model variable to a single value.  As
\cite{BreunessePoll03} points out, if there are no possible values satisfying the predicate, 
inconsistency in the generated axioms can result, if appropriate care is not taken.

In this case, we represent the model variable by a method call, with state and receiver arguments
as discussed above.

\subsubsection{Model variables with no representation}

A model variable, particularly in an abstract class or interface, may have no representation
at all.  It may be used in the specification of various methods, but its representation would be
supplied by derived classes that implement the interface.  An example of such an interface is
in Fig.~\ref{fig:norep}.  In this situation, we still translate the model variables as method calls,
but now there are no axioms generated from the represents clauses.  Instead, only the
pre- and post-conditions provide information about the behavior of the variable.

\begin{BFIGURE}
\verbatiminput{NoRep.java}
\caption{The specification and code for the interface \texttt{NoRep},
demonstrating a model variable with no representation.}
\label{fig:norep}
\end{BFIGURE}

Representation-less model variables do pose a difficulty in the translation and checking of model bodies.  If the model variable is \texttt{assignable} for a given method, then its value may 
change in the course of execution of the body of that method.  But without a representation,
there is no way to reason about its value during any (symbolic) execution of the 
body.~\footnote{The requirements on JML's  \texttt{in} clause, not discussed here, do provide information
on which other fields of the class contribute to the value of a model variable and consequently
at which program points the value of a model variable might change.}

\subsubsection{Unmodified model variables}

There is a special case of using model variables that allows a simplification in their translation.
If a model variable is not listed~\footnote{including not listed via a JML datagroup} in a
method's \texttt{assignable} clause, then its value may not change during the course of the
body of the method.  Thus for that method's body and the checking of its pre- and
post-conditions, the model variable may be treated as a constant.  The value of the model 
variable can be constrained by the value of its representation evaluated in the pre-state
(or any state) of the method.

\section{Outstanding issues in using \\ method calls in annotations}

Although the translation for method calls described above is implemented with 
favorable experience in verification, there remain some open issues and opportunities 
for improvement.

\subsection{Immutable values}

The complication of introducing state constants as additional arguments is a result of the underlying logic using uninterpreted values for reference quantities in the programming language.
Since these reference values refer to mutable objects, one must retain a state value to 
indicate which state of the object is meant.  If all of the arguments were primitive type values
such as integers and booleans, then a state value would not be needed.  These values of
non-reference types are immutable: if two values compare equal, they will always have the same
properties in any program state. 

Some reference types are also immutable.  For example, values of \texttt{java.lang.String} that
compare equal (with \texttt{==}) will always have identical properties even in different program
states.  One requirement for the values of a type to be immutable is that no method of the type
modify the internal state of the object; this condition is assured to hold if all methods of the type are
\texttt{pure} and all methods of any subtype are \texttt{pure}.  However, it is also at least
necessary for immutability
that the internal representation not contain mutable objects and
that the representation not be exposed  in a way that the internal state could be modified by 
some external means.

Reasoning with immutable objects is potentially simpler and more efficient that with typical
mutable objects.  For example, JML includes a library of classes representing mathematical 
concepts useful in specifying classes \cite{Leavens-etal03a}; 
it may be advantageous if it could be 
established that values of these classes are immutable.  For this to be possible, we need 
a set of sufficient conditions for immutability that can be statically checked, a proof of 
soundness regarding immutability, and a demonstration by a working implementation
of the utility of immutable classes in program verification .

%%% No room for a discussion of constancy

\subsection{Exceptions}

The evaluation of the expression in a model variable's represents clause might result in
exceptional behavior (that is, in the throwing of a Java exception).  In an environment
where the truth or falsity of assertions must be determined by the reasoning engine, a
semantics for specifications with exceptional termination must be established.  For example,
what should be the behavior of a verification system on encountering the program of 
Fig.~\ref{fig:exc}?  Consider just the precondition of method \texttt{p}: should the program 
checker issue a warning that there is a potential dereference of a null reference, or should it
simply presume an implicit precondition that \texttt{o != null}?

\begin{BFIGURE}
\verbatiminput{NoRep.java}
\caption{The specification and code for a class in which methods used in specifications throw exceptions.}
\label{fig:norep}
\end{BFIGURE}

Various approaches to handling exceptional expressions are described in previous 
publications.
In \ref{Cheon-Leavens02b}, Cheon takes a contextual approach appropriate to run-time 
assertion checking.   In \ref{Leavens-Baker-Ruby02}, it is proposed that 
an arbitrary, unknown value be substituted for an expression that terminates abruptly.
This has the effect (perhaps surprising to the Java programmer) that
\texttt{(o == null || o.a == 0)} and \texttt{(o.a == 0 || o == null)} have the same boolean value.
The LOOP project formalized exceptions differently \cite{BergPJ00a}, in the end recommending
that operators such as \texttt{||} be interpreted precisely as in Java and that specifications be
fully guarded so that no exceptional termination will occur.

The choice of semantics for exceptional expressions obviously affects the tools that would
implement any static checking, but also affects the convenience of reading and 
writing specifications and the success of achieving the goal of defining specifications (in JML, 
at least) so that they are unsurprising to Java programmers.  ESC/Java2 currently has
undefined behavior if a specification expression might terminate abruptly.


\section{Background: Related work}

There are by now a number of research tools that translate Java into verification conditions
appropriate to some logical reasoning engine.  There are numerous tools that insert
assertion checking code into compiled Java programs, and others that do model checking,
but
we are aware only of the following tools that do static
checking of assertions:
\begin{itemize}
\setlength{\itemsep}{0in}
\item Krakatoa \cite{krakatoa03a} uses the Coq prover and 
translates all statements with reference to a current program state. 
\item The LOOP tool \cite{JacobsPoll01a,Jacobs03}
 translates Java using weakest-preconditions in a Hoare logic,
which are then proved using PVS \cite{OwreRRSS96}, interactively.  
\item The ChAsE tool \cite{CH03} is an extension of ESC/Java that checks frame conditions.
\item Jive \cite{MeyerMuellerPoetzsch-Heffter00} translates a subset of Java into proof
obligations for PVS; it supplies an interactive tool for managing the proofs.
\item JACK \cite{JACK} generates proof conditions directly for the Atelier B prover; 
most, but not all, of
the proof obligations can be handled automatically. 
\end{itemize}

ESC/Java2, Krakatoa and ChAsE are made publicly available.  Only ESC/Java2 targets a
generic first-order prover and seeks to be completely automatic.  On the other hand, ESC/Java
did not intend to be sound or complete, but made some engineering compromises in order to
be automatic and reasonably useful to programmers not familiar with proof techiques or
interactive proof tools.  The approaches taken by the Krakatoa and LOOP tools in mapping
Java into the underlying logic are the most well-documented.

ESC/Java2 does not utilize a full state-based logic, employing program states only where needed.
Krakatoa uses a reference to the state of the heap throughout its translation; though this is an
accurate model of Java semantics, it can be complex to use in actual proofs.  The separation of
values in ESC/Java2 by using separate arrays for each field helps to automatically resolve some
aliasing problems.

The LOOP tool utilizes a special-purpose logic appropriate for translating imperative programming
langauges.  The representation of a program is then converted into PVS theories and proved
using the PVS theorem prover.

Though there are similar aspects among these approaches, the solution used by ESC/Java2
for translating method calls demonstrates a straightforward translation in the context of 
a general purpose logic and prover.  In doing so it maintains the design philosophy and usefulness
of the original ESC/Java tool.

\section{Acknowledgments}

Joseph Kiniry is supported by the NWO Pionier research
project on Program Security and Correctness and the VerifiCard
research project. 


%
% The following two commands are all you need in the
% initial runs of your .tex file to
% produce the bibliography for the citations in your paper.
\bibliographystyle{abbrv}
%\bibliographystyle{plain}

\bibliography{PASTE2004}  
% To create the self-contained file - comment out the line above and include the
% contents of the .bbl file here


% You must have a proper ".bib" file
%  and remember to run:
% latex bibtex latex latex
% to resolve all references
%
% ACM needs 'a single self-contained file'!
%
%APPENDICES are optional

%\balancecolumns


% That's all folks!
\end{document}
