\documentclass[
%a4,
nocolorBG,
%colorBG,
slideColor,
pdf,
%accumulate,ps,
%draft,
erik,
%frames,
]{prosper}
%%%%%%%%%%%%%%%%%%%%%%%%%%%%%%%%%%%%%%%%%%%%%%%%%%%%%%%%%%%%%%%%%%%%%%%%%%%
\usepackage{alltt}
\usepackage{pstricks,pst-node,pst-text,pst-3d}
\usepackage{textcomp}
%\usepackage{colordvi}
\newcommand{\Red}[1]{{\red #1}}
\usepackage[all]{xy}

%%%%%%%%%%%%%%%%%%%%%%%%%%%%%%%%%%%%%%%%%%%%%%%%%%%%%%%%%%%%%%%%%%%%%%%%%%%

%\newrgbcolor{Yellowish}{0.90 0.85 0.650}

%\newrgbcolor{red}{1 0 0}
%\newrgbcolor{purple}{0.4 0 0.7}
%\newrgbcolor{lightpurple}{0.63 0.13 0.94}

%\newrgbcolor{lime}{0.73 1 0}
\newrgbcolor{green}{0.133 0.56 0} % lichter
%\newrgbcolor{green}{0.10 0.43 0}

%\newrgbcolor{knalblue}{0 0 1}
\newrgbcolor{blue}{.2 .36 .77}
%\newrgbcolor{darkblue}{0.28 0.24 0.55}


%%%%%%%%%%%%%%%%%%%%%%%%%%%%%%%%%%%%%%%%%%%%%%%%%%%%%%%%%%%%%%%%%%%%%%%%%%%

\newcommand{\embf}[1]{\textit{\textbf{#1}}}
\newcommand{\rmbf}[1]{\textrm{\textbf{#1}}}
\newcommand{\ttbf}[1]{\mbox{\tt \textbf{#1}}}
\newcommand{\code}[1]{{\rm \texttt{\textbf{\small #1}}}}



\myitem{1}{\mbox{{$\bullet$}}}
\newcommand{\old}{\(\backslash\)old}
\newcommand{\vooralle}{\(\backslash\)forall}
\newcommand{\result}{\(\backslash\)result}
\renewcommand{\familydefault}{phv}
\renewcommand{\rmdefault}{phv}
\newif\ifignore

\newrgbcolor{Bluish}{0.9 0.9 1.0}
\newcommand{\doos}[1]{\psshadowbox[fillstyle=solid,
                        fillcolor=Bluish,
                        framearc=0.2,
                        framesep=2mm]
                        {#1}}

%%%%%%%%%%%%%%%%%%%%%%%%%%%%%%%%%%%%%%%%%%%%%%%%%%%%%%%%%%%%%%%%%%%%%%%%%%%%%

\title{\embf{\blue 
       {\huge ESC/Java 
       \\ \medskip
       {\Large extended static checking for Java }}
      }}
\author{\embf{\Large{\red Erik Poll}}
       }
\institution{\embf{\large University of Nijmegen}}
\slideCaption{{\blue Erik Poll - JML}}
%\Logo{{\blue Erik Poll}}

%%%%%%%%%%%%%%%%%%%%%%%%%%%%%%%%%%%%%%%%%%%%%%%%%%%%%%%%%%%%%%%%%%%%%%%%%%%%%

\begin{document}

\maketitle 

\boldmath


%%%%%%%%%%%%%%%%%%%%%%%%%%%%%%%%%%%%%%%%%%%%%%%%%%%%%%%%%%%%%%%%%%%%%%%%%%%%%
\overlays{4}{
\begin{slide}{ESC/Java}
\vspace*{-3ex}

{\blue Extended static checker}
{\rm by Rustan Leino et.al. [Compaq].}
\begin{itemstep}
\item$\!$
{\blue {\em tries} to {\em prove} correctness of specifications,}\\
{\green at compile-time, fully automatically}
\item$\!$
{\em not sound}, {\em not complete}, but {\blue finds lots of potential bugs quickly}
\item
good at proving absence of runtime exceptions
{\scriptsize (eg Null-, ArrayIndexOutOfBounds-, ClassCast-)}
and verifying relatively simple properties.
\item
ESC/Java only supported a subset of full JML, but
ESC/Java2 by Joe Kiniry [KUN] \& David Cok [Kodak] remedies this.
\end{itemstep}

\end{slide}}

%%%%%%%%%%%%%%%%%%%%%%%%%%%%%%%%%%%%%%%%%%%%%%%%%%%%%%%%%%%%%%%%%%%%%%%%%%%%
\begin{slide}{static checking vs runtime checking}
\vspace*{-3ex}

Important differences:

\begin{itemize}
\item ESC/Java checks specs at {\blue compile-time}, \\
      jmlc checks specs at {\green run-time}
\item ESC/Java {\blue proves} correctness of specs,\\
      jml only {\green tests} correctness of specs.
\\
Hence
\begin{itemize}
\item ESC/Java independent of any test suite, \\
      results of runtime testing only as good as the test suite,
\item ESC/Java provided higher degree of confidence.
\end{itemize}

\end{itemize}

\end{slide}

%%%%%%%%%%%%%%%%%%%%%%%%%%%%%%%%%%%%%%%%%%%%%%%%%%%%%%%%%%%%%%%%%%%%%%%%%%%%
\begin{slide}{ESC/Java ``demo''}
\vspace*{-6ex}
\begin{alltt}
\texttt{\textbf{\footnotesize
class Bag \{
  int[] a;
  int   n;
  int{\green extractMin}() \{
   int m = Integer.MAX_VALUE;
   int mindex = 0;
   for (int i = 1; i <= n; i++) \{
        if (a[i] < m) \{ mindex =i; m = a[i]; \} \}
   n--;
   a[mindex] = a[n];
   return m;
 \} 
}}
\end{alltt} %}
\end{slide}



%%%%%%%%%%%%%%%%%%%%%%%%%%%%%%%%%%%%%%%%%%%%%%%%%%%%%%%%%%%%%%%%%%%%%%%%%%%%
\begin{slide}{ESC/Java ``demo''}
\vspace*{-6ex}
\begin{alltt}
\texttt{\textbf{\footnotesize
class Bag \{
  int[] a;
  int   n;
  int{\green extractMin}() \{
   int m = Integer.MAX_VALUE;
   int mindex = 0;
   for (int i = 1; i <= n; i++) \{
        if (\Red{a[i]} < m) \{ mindex =i; m = a[i]; \} \}
   n--;
   a[mindex] = a[n];
   return m;
 \}
}}
\end{alltt} %}
\vspace*{-3ex}
\doos{Warning: {\red possible null deference}. Plus other warnings}
\end{slide}


%%%%%%%%%%%%%%%%%%%%%%%%%%%%%%%%%%%%%%%%%%%%%%%%%%%%%%%%%%%%%%%%%%%%%%%%%%%%
\begin{slide}{ESC/Java ``demo''}
\vspace*{-6ex}
\begin{alltt}
\texttt{\textbf{\footnotesize
class Bag \{
  int[] a; {\blue //@ invariant a != null;}
  int   n;
  int{\green extractMin}() \{ 
   int m = Integer.MAX_VALUE;
   int mindex = 0;
   for (int i = 1; i <= n; i++) \{
        if (a[i] < m) \{ mindex =i; m = a[i]; \} \}
   n--;
   a[mindex] = a[n];
   return m;
 \}
}}
\end{alltt} %}
\vspace*{-3ex}
\end{slide}
 
%%%%%%%%%%%%%%%%%%%%%%%%%%%%%%%%%%%%%%%%%%%%%%%%%%%%%%%%%%%%%%%%%%%%%%%%%%%%
\begin{slide}{ESC/Java ``demo''}
\vspace*{-6ex}
\begin{alltt}
\texttt{\textbf{\footnotesize
class Bag \{
  int[] a; {\blue //@ invariant a != null;}
  int   n;
  int{\green extractMin}() \{ 
   int m = Integer.MAX_VALUE;
   int mindex = 0;
   for (int i = 1; i <= n; i++) \{
        if (\Red{a[i]} < m) \{ mindex =i; m = a[i]; \} \}
   n--;
   a[mindex] = a[n];
   return m;
 \}
}}
\end{alltt} %}
\vspace*{-3ex}
\doos{Warning: {\red  Array index possibly too large }}
\end{slide}




%%%%%%%%%%%%%%%%%%%%%%%%%%%%%%%%%%%%%%%%%%%%%%%%%%%%%%%%%%%%%%%%%%%%%%%%%%%%
\begin{slide}{ESC/Java ``demo''}
\vspace*{-6ex}
\begin{alltt}
\texttt{\textbf{\footnotesize
class Bag \{
  int[] a; {\blue //@ invariant a != null;}
  int   n; {\blue //@ invariant 0 <= n && n <= a.length;}
  int{\green extractMin}() \{
   int m = Integer.MAX_VALUE;
   int mindex = 0;
   for (int i = 1; i <= n; i++) \{
        if (a[i] < m) \{ mindex =i; m = a[i]; \} \}
   n--;
   a[mindex] = a[n];
   return m;
 \}
}}
\end{alltt} %}
\vspace*{-3ex}
\end{slide}

%%%%%%%%%%%%%%%%%%%%%%%%%%%%%%%%%%%%%%%%%%%%%%%%%%%%%%%%%%%%%%%%%%%%%%%%%%%%
\begin{slide}{ESC/Java ``demo''}
\vspace*{-6ex}
\begin{alltt}
\texttt{\textbf{\footnotesize
class Bag \{
  int[] a; {\blue //@ invariant a != null;}
  int   n; {\blue //@ invariant 0 <= n && n <= a.length;}
  int{\green extractMin}() \{
   int m = Integer.MAX_VALUE;
   int mindex = 0;
   for (int i = 1; i <= n; i++) \{
        if (\Red{a[i]} < m) \{ mindex =i; m = a[i]; \} \}
   n--;
   a[mindex] = a[n];
   return m;
 \}
}}
\end{alltt} %}
\vspace*{-3ex}
\doos{Warning: {\red  Array index possibly too large }}
\end{slide}


%%%%%%%%%%%%%%%%%%%%%%%%%%%%%%%%%%%%%%%%%%%%%%%%%%%%%%%%%%%%%%%%%%%%%%%%%%%%
\begin{slide}{ESC/Java ``demo''}
\vspace*{-6ex}
\begin{alltt}
\texttt{\textbf{\footnotesize
class Bag \{
  int[] a; {\blue //@ a != null; }
  int   n; {\blue //@ invariant 0 <= n && n <= a.length;}
  int{\green extractMin}() \{
   int m = Integer.MAX_VALUE;
   int mindex = 0;
   for (int i = \Red{0}; i \Red{<} n; i++) \{
        if (a[i] < m) \{ mindex =i; m = a[i]; \} \}
   n--;
   a[mindex] = a[n];
   return m;
 \}
}}
\end{alltt} %}
\end{slide}

%%%%%%%%%%%%%%%%%%%%%%%%%%%%%%%%%%%%%%%%%%%%%%%%%%%%%%%%%%%%%%%%%%%%%%%%%%%%
\begin{slide}{ESC/Java ``demo''}
\vspace*{-6ex}
\begin{alltt}
\texttt{\textbf{\footnotesize
class Bag \{
  int[] a; {\blue //@ a != null; }
  int   n; {\blue //@ invariant 0 <= n && n <= a.length;}
  int{\green extractMin}() \{
   int m = Integer.MAX_VALUE;
   int mindex = 0;
   for (int i = 0; i < n; i++) \{
        if (a[i] < m) \{ mindex =i; m = a[i]; \} \}
   n--;
   a[mindex] = \Red{a[n]};
   return m;
 \}
}}
\end{alltt} %}
\vspace*{-3ex}
\doos{Warning: {\red  Possible negative array index }}
\end{slide}


%%%%%%%%%%%%%%%%%%%%%%%%%%%%%%%%%%%%%%%%%%%%%%%%%%%%%%%%%%%%%%%%%%%%%%%%%%%%
\begin{slide}{ESC/Java ``demo''}
\vspace*{-7ex}
\begin{alltt}
\texttt{\textbf{\footnotesize
class Bag \{
  int[] a; {\blue //@ a != null; }
  int   n; {\blue //@ invariant 0 <= n && n <= a.length;}
 {\blue //@ requires n > 0;}
  int{\green extractMin}() \{
   int m = Integer.MAX_VALUE;
   int mindex = 0;
   for (int i = 0; i < n; i++) \{
        if (a[i] < m) \{ mindex =i; m = a[i]; \} \}
   n--;
   a[mindex] = a[n];
   return m;
 \}
}}
\end{alltt} %}
\end{slide}

%%%%%%%%%%%%%%%%%%%%%%%%%%%%%%%%%%%%%%%%%%%%%%%%%%%%%%%%%%%%%%%%%%%%%%%%%%%%
\begin{slide}{ESC/Java ``demo''}
\vspace*{-7ex}
\begin{alltt}
\texttt{\textbf{\footnotesize
class Bag \{
  int[] a; {\blue //@ a != null; }
  int   n; {\blue //@ invariant 0 <= n && n <= a.length;}
 {\blue //@ requires n > 0;}
  int{\green extractMin}() \{
   int m = Integer.MAX_VALUE;
   int mindex = 0;
   for (int i = 0; i < n; i++) \{
        if (a[i] < m) \{ mindex =i; m = a[i]; \} \}
   n--;
   a[mindex] = a[n];
   return m;
 \}
}}
\end{alltt} %}
\vspace*{-4ex}
\doos{No more warnings about this code}
\end{slide}

%%%%%%%%%%%%%%%%%%%%%%%%%%%%%%%%%%%%%%%%%%%%%%%%%%%%%%%%%%%%%%%%%%%%%%%%%%%%
\begin{slide}{ESC/Java ``demo''}
\vspace*{-7ex}
\begin{alltt}
\texttt{\textbf{\footnotesize
class Bag \{
  int[] a; {\blue //@ a != null; }
  int   n; {\blue //@ invariant 0 <= n && n <= a.length;}
 {\blue //@ requires n > 0;}
  int{\green extractMin}() \{
   int m = Integer.MAX_VALUE;
   int mindex = 0;
   for (int i = 0; i < n; i++) \{
        if (a[i] < m) \{ mindex =i; m = a[i]; \} \}
   n--;
   a[mindex] = a[n];
   return m;
 \}
}}
\end{alltt} %}
\vspace*{-4ex}
\doos{\parbox{24em}{\ldots but warnings about calls to \code{extractMin()} that do not ensure precondition}}

\end{slide}


%%%%%%%%%%%%%%%%%%%%%%%%%%%%%%%%%%%%%%%%%%%%%%%%%%%%%%%%%%%%%%%%%%%%%%%%%%%%
\overlays{3}{
\begin{slide}{Some points to note}
\vspace*{-3ex}
\begin{itemstep}
\item
ESC/Java {\em {\green forces}}  to specify some properties.
\medskip
\item
If you inderstand the code, \\
then these properties are obvious.
\medskip

{\em But for larger programs this may not be the case!}
\medskip
\item
If you have these properties documented, \\
then understanding the code is easier.
\end{itemstep}
\end{slide}}

%%%%%%%%%%%%%%%%%%%%%%%%%%%%%%%%%%%%%%%%%%%%%%%%%%%%%%%%%%%%%%%%%%%%%%%%%%%%
\begin{slide}{ESC/Java vs runtime checking (cont.)}
\vspace*{-3ex}
\begin{itemize}
\item
For {\green runtime assertion checking}, we could {\green choose
what we specify}, e.g. all, one, or none of the properties
we have written for for \code{Bag}.
\item
But for {\blue ESC/Java} to accept a spec, we are
{\em {\blue forced}}  to specify {\em {\blue all properties}}
(e.g.\ invariants, preconditions)
that this spec relies on.
\end{itemize}
\end{slide}

%%%%%%%%%%%%%%%%%%%%%%%%%%%%%%%%%%%%%%%%%%%%%%%%%%%%%%%%%%%%%%%%%%%%%%%%%%%%%
\begin{slide}{Limitations of ESC/Java}
\vspace*{-3ex}

ESC/Java is
\begin{itemize}
\item {\blue not sound}: it may {\green complain a correct spec}
\item {\blue not complete}: it may {\green accept an incorrect spec}
\end{itemize}
ESC/Java warns about {\green many potential bugs} , but {\green not about all actual bugs.}

\medskip
These are unavoidable concessions to main goal:\\
{\green pointing out lots of potential bugs quickly \& completely automatically}

\medskip

In practice ESC/Java is quite good at checking
simple specs, e.g.\ ruling out any
\code{NullPointer-}
and \code{IndexOutOfBounds\-Exceptions}

\end{slide}

\end{document}

